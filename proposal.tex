\documentclass[amstex,11pt]{amsart}
\usepackage{datetime}
\usepackage{amsmath}
\usepackage{graphics}
\usepackage{verbatim}

\usepackage{url}
\usepackage{amssymb}
\usepackage{mathrsfs}
\usepackage{cite}
\DeclareMathOperator{\SL}{SL}
\DeclareMathOperator{\GL}{GL}
\DeclareMathOperator{\PGL}{PGL}
\DeclareMathOperator{\Sp}{Sp}
\DeclareMathOperator{\GSp}{GSp}
\newcommand{\pp}{\mathfrak{p}}
\newcommand{\nn}{\mathfrak{n}}
\newcommand{\p}{\mathfrak{p}}
\newcommand{\n}{\mathfrak{n}}
\newcommand{\ncisom}{\approx}   % noncanonical isomorphism


% Linked CONTENTS
\usepackage{hyperref}
%%%%%%%%%%%%%%%%%

%\setlength{\parskip}{2ex}
\setlength{\parskip}{0.1ex}

\voffset -0.5in
\hoffset -0.8in
%\oddsidemargin  -0.3125in
%\evensidemargin -0.3125in
\textwidth      6.475in
\headheight     0.0in
\topmargin      0.063in
\textheight=9.3875in

\numberwithin{equation}{section}

% ---- SHA ----
%\DeclareFontEncoding{OT2}{}{} % to enable usage of cyrillic fonts
%  \newcommand{\textcyr}[1]{%
%    {\fontencoding{OT2}\fontfamily{wncyr}\fontseries{m}\fontshape{n}%
%     \selectfont #1}}
%\newcommand{\Sha}{{\mbox{\textcyr{Sh}}}}
%\newcommand{\cyr}{\textcyr}
\newcommand{\nc}{\newcommand}
\newcommand{\C}{\mathbb C}
\newcommand{\Q}{\mathbb Q}
\newcommand{\QQ}{\mathbb Q}
\newcommand{\PP}{\mathbb P}
\newcommand{\Z}{\mathbb Z}
\newcommand{\R}{\mathbb R}
\newcommand{\gal}{\textrm{Gal}}

% from PG
\newcommand{\bG}{\mathbf{G}}
\newcommand{\sO}{\mathscr O}
\newcommand{\sM}{\mathscr M}
\newcommand{\bP}{\mathbf{P}}
\newcommand{\F}{\mathbb{F}}
\newcommand{\fH}{\mathfrak{H}}
\newcommand{\cusp}{\text{cusp}}

\title{Conference: Curves and Automorphic Forms}
\begin{document}
%\begin{center}
%   \settimeformat{ampmtime}\bf \Large \today{} at \currenttime{}
%\end{center}
%(Temporary table of contents so we can find our way around.)
%\tableofcontents\vfill\mbox{}
%
%\break


\section{Overview}

The theme for this conference is computational aspects of connections
between curves and modular forms.  Our goals are to provide a setting
to
\begin{itemize}
\item allow dissemination of relevant recent research results
\item foster collaboration among experts working in areas which are
  related, or are conjecturally related
\item aid in the development of number theorists who are early in
  their careers and/or from underrepresented groups
\item make relevant computational results publicly available.
\end{itemize}

The format of the event will combine elements of conferences and
workshops to achieve its goals.  Each morning will feature two plenary
talks given by invited speakers starting with background for their
area and leading to recent results.  Afternoons will be devoted to
participants working in small groups.  Some groups will work on
questions raised by the morning lectures, others will work on
computations, and yet others will work on presenting computational
results to the public.

Computational results will be presented through the web site {\em
  $L$-functions and Modular Forms Database}
(\url{http://www.LMFDB.org}), which we will refer to as the LMFDB.
This site is an international collaboration dedicated to presenting
computational data in number theory in a way which will assist
mathematicians conducting research, and also in a way to help educate
visitors to the site on central objects of modern number theory, and their
interconnections.

Computational projects provide a good path for young mathematicians to
start to get a feel for an area.  Given the topic, most of the senior
participants will be well-versed in doing computations in number
theory.  We will have experts from the LMFDB project in attendance to
aid people who may be contributing to that project for the first
time.  Since the LMFDB uses Sage, an open-source computational system,
we invite a significant number of participants from the {\em Women in
  Sage} community to help broaden the representation of women in this
conference.

\bigskip

Theoretical and computational work on elliptic curves over the
rationals has enjoyed great advances over the past few decades.
Recently there has been surge of work on extending this in two
directions, namely to hyperelliptic curves over the rationals and
elliptic curves over quadratic fields.  In additional, there have been
parallel advances on the corresponding automorphic objects, namely
Siegel and Hilbert modular forms.  A major objective of this
conference is to present this recent work to a wider audience and in
the process clarify the connections between the algebraic and
automorphic objects.  There is a large amount of useful related
numerical data which we would like to make available to the
mathematics research community.

Recent advances to be covered in workshop lectures:
\begin{enumerate}
\item  Elliptic curves/$\Q(\sqrt{5})$.

\noindent
William Stein and collaborators have been developing the analogue of
Cremona's tables for elliptic curves over $\Q(\sqrt{5})$.  There are
numerous difficulties which are not present in the rational case:
modularity is not known, given the modular form, the construction of
the corresponding elliptic curve is not known, etc.

%One or two lectures by Stein.

\item Hilbert modular forms

\noindent
John Voight has made extensive tables of Hilbert modular forms,
primarily over $\Q(\sqrt{5})$ and $\Q(\sqrt{8})$.  The basic
theory, the computational methods, and the associated $L$-functions
(which provide the connection to elliptic curves) will be
described.

%One or two lectures by Voight.

\item  Hyperelliptic curves

\noindent
Kiran Kedlaya and Andrew Sutherland have produced a wide variety
of examples of hyperelliptic curves (primarily genus 2, but genus 3
calculations will be available by the time of the workshop).

%One lecture each by Sutherland and Kedlaya

\item  Paramodular forms

\noindent
Siegel modular forms on the paramodular group are (conjecturally) the
modular objects associated to genus 2 hyperelliptic curves.
Cris Poor and David Yuen have produced the first examples of these
objects.  These calculations are significantly more difficult than
the corresponding work on Hilbert modular forms.

%One lecture by Poor or Yuen

\item  Elliptic curves/$\Q(i)$

\noindent
Elliptic curves over imaginary quadratic fields are fundamentally
different than elliptic curves over real quadratic fields, primarily
because the associated modular object is a Bianchi modular form,
not a Hilbert modular form.  Work in this area is much less advanced
(despite the fact that John Cremona made tables of elliptic curves
over $\Q(i)$ in his thesis.

%One lecture by Cremona

\item  $L$-function techniques for hyperelliptic curves

\noindent
Current methods of producing tables of hyperelliptic curves are not
capable of proving that the list is complete.  David Farmer, Sally
Koutsoliotas, and Stefan Lemurell and an $L$-function approach which
(assuming modularity) can provide the missing step in the proof.

%One lecture by Koutsoliotas and/or Lemurell
\end{enumerate}

\section{Scientific content}

\subsection{Hyperelliptic curves over $\Q$}

In this workshop we will focus on recent work concerning hyperelliptic 
curves
of genus~$2$, that is, a smooth curve of the form
\begin{equation}
y^2=f(x)
\end{equation}
where $f\in k[x]$ is a polynomial of degree 5 or~6, where $k/\Q$ is a 
number field.
Our specific focus is on the recent work of Kedlaya and Sutherland on the
distribution of Euler factors of the $L$-function of the curve.  This work
concerns abelian surfaces; at the workshop we will bring out the 
connections
with other objects.

It is not the hyperelliptic curve, but rather its jacobian to which we
can associate an $L$-function.  The jacobian, $A$, is an abelian 
surface/$k$.
What Kedlaya and Sutherland proved is that there are (up to conjugacy)
52 subgroups of $USp(4)$ which determine the distribution of Euler factors
of the $L$-function of $A$.  That is, for each abelian surface $A$ there is
a closed subgroup of $USp(4)$, called the Sato-Tate group of $A$,
such that the local factors of the
$L$-function of $A$ have the same limiting distribution as the 
characteristic
polynomials of matrices in the subgroup.
They exhibit hyperelliptic curves whose Jacobians give examples of all 52 
cases,
of which 34 occur for $k=\Q$.

Workshop lectures will describe how the Sato-Tate group arises
from the Galois action on the Tate module of~$A$, and briefly describe how
the subgroups were classified.  While some of this material can be 
described
for any number field, we will primarily consider hyperelliptic curves over 
the
rationals.  In this case the $L$-function has degree~4, which makes 
computation
more tractable, and there are connections to objects from other lectures,
such as Siegel modular forms.  The role of the Sata-Tate group in 
identifying
the associated modular object, and determining its properties, will be the
starting point for discussions on future avenues of research.



%\subsection{Elliptic curves over totally real number fields}

\subsection{Elliptic curves over low degree number fields}

An elliptic curve over a number field $k/\Q$ has a $L$-function of
degree~$2d$, where $d=[k:\Q]$.  We will be primarily concerned with
quadratic fields, because of their relation to other objects already
in the LMFDB.

In the case of real quadratic fields, lectures will focus on the
computational challenges faced by Stein and his collaborators as they
tabulate elliptic curves over $\Q(\sqrt5)$.  There are numerous
aspects of that work which are hampered by our rudimentary knowledge
of this area; this discussion will be the starting point for research
project involving both theory and computation.

The modular object associated to $E/\Q(\sqrt5)$ is a Hilbert modular
form on $\GL(2)/\Q(\sqrt5)$, which will be the topic of another
series of lectures.


\subsection{Hilbert modular forms}

\subsection{Siegel modular forms}

Computation of Siegel modular forms is quite rudimentary compared to
the computation of either classical or Hilbert modular forms.  For
higher weight there is the possibility of employing cohomological
methods (the analogue os modular symbols for classical holomorphic
modular forms).  However, for the weight 2 case, which is associated
to hyperelliptic curves, this is not an option.  Specifically, we are
interested in weight 2 cusp forms on the level-$N$ paramodular group,
defined as
$$
K(N)=\left\{
\left(\begin{matrix}
  * & N* & * &*\\
* & * & * & */N\\
* & N* &*&*\\
N*&N*&N*&*
\end{matrix}\right)
:
*\in{\Bbb Z}\right\}\cap \mathrm{Sp}(4, {\Bbb Q}).
$$
This group is an an analogue of the Hecke congruence group
$\Gamma_0(N)$, in the sense that it is normalized by an analogue of
the Fricke involution.

Current methods, primarily due to Poor and Yuen, are computationally
intensive due to the difficulty of finding a basis for the space of
cusp forms.  The main technique is via descent from higher weights,
and exploiting the ``smearing'' action of Hecke operators acting on
non-cuspforms.  Unless there are new developments between now and the
workshop, lectures on Siegel modular forms will briefly cover
background and computational methods, and then focus on the
``paramodular conjecture,'' which is the higher rank analogue of
modularity for elliptic curves.


\subsection{$L$-functions}

An $L$-function is a Dirichlet series with a functional equation and
an Euler product, plus a few technical conditions.  One view of
$L$-functions is that they are the glue which connects related
mathematical objects.  For example, a geometric object is said to be
modular if it has the same $L$-function as an automorphic form.

In this workshop the general theory of $L$-functions will be
introduced, and this framework will be used to provide a high-level
description of the connections between the other objects which have
been introduced.  There are two topics of current research which will
be presented and then will form the basis of later discussions.  The
first is the recent discovery by Farmer, Koutsoliotas, and Lemurell
that it is possible in many cases to generate the $L$-function of an
object without first finding the object itself.  This method has been
successful for a variety of degree~4 $L$-functions, but it does not
work in all cases.  Since the objects being described in the other
lectures are associated with degree 4 $L$-functions, this topic is
timely and the methods and current limitations will be described
carefully.

The second topic is the fact that high degree $L$-functions (degree 3
or larger) are computationally expensive to evaluate.  This limits our
ability to make detailed studies of, for example, the distribution of
zeros of these functions.  The underlying cause of these computational
problems will be described, leading to discussions for possible ways
to improve the calculations.  One possibility, which is speculative
but worth considering, is that properties of the underlying objects
could be exploited in some way.  This will take advantage of the fact
that both the $L$-functions and their underlying objects will be
discussed in detail at the workshop.


\begin{comment}

Among the class of algebraic curves, hyperelliptic curves on one hand
form the most easily studied subclass due to the explicit nature of
their description, and on the other hand are general enough to exhibit
many phenomena typical of arbitrary algebraic curves.
%The hyperelliptic curves of genus 1 are exactly the elliptic
%curves and their nontrivial torsors; we thus focus in
%this discussion on curves of genus greater than 1.

Achieving a meaningful and useful classification of hyperelliptic
curves in the LMFDB, comparable to the Cremona database of elliptic
curves, would require progress in the following
directions. (Throughout this discussion, we will limit attention to
hyperelliptic curves over $\QQ$, but we may consider working over
other number fields later.)

\begin{enumerate}
\item {\em Tabulation of Euler factors at primes of good reduction.}
  For genus 2 and 3, this is largely addressed by the work of Kedlaya
  and Sutherland \cite{kedlaya-sutherland-ants}, \cite{smalljac},
  which incorporates $p$-adic cohomology in the form of an algorithm
  of Harvey \cite{harvey} along with a highly optimized baby-step
  giant-step algorithm for smaller primes.
%When the Jacobian of the curve is modular, we can also use modular symbols.

\item\label{hyper:euler} {\em Computation of Euler factors and other
    associated data at primes of bad reduction, e.g., the conductor.}
  This is most tractable in genus 2, where an implementation of an
  algorithm due to Qing Liu (which computes the conductor and
  reduction type for odd residue characteristic) is available in Sage.
  Extending this to the case of residue characteristic 2 is not
  straightforward, but an algorithm for computing regular models is
  implemented in Magma and another one is currently being implemented
  in Singular.  Using this, one can find the Euler factors using the
  Galois action on the dual graphs of the special fibers of the
  regular model and the Euler factors for components of positive genus
  in those special fibers. To find the conductor one needs a
  semistable model. Algorithms for computing this and deducing the
  conductor are currently being

developed by Wewers.  Alternatively, one can conditionally collect
some of the same data using analytic methods, e.g., methods of Booker
\cite{Bo2} or generic methods using the approximate functional
equation (see Section~\ref{sec:findingbadfactors}).


\item
{\em Computation of Mordell-Weil groups of Jacobians of hyperelliptic
curves.} Computing generators of a subgroup of finite index may be
facilitated by the recent work of Bhargava and Gross on 2-Selmer
groups of hyperelliptic curves \cite{bhargava-gross}, but some work
(suitable for a graduate student or postdoc) is still needed.  In
order to compute generators for the full Mordell-Weil group, one needs
to saturate a given finite index subgroup. For $g=2$, everything has
been implemented in Magma; for $g=3$, this is addressed by work in
progress of Stoll. For arbitrary $g$, canonical heights can be
computed using the algorithm from \cite{mueller:computing}
(implemented in Magma). In addition, one needs to bound the difference
between canonical and naive heights and search for points of naive
height up to a given bound.  Recent work of Holmes
%\cite{holmes:arakelov-height} gives an algorithm for both of these,
\cite{holmes:arakelov-height} provides such an algorithm,
but to make it practical the resulting bounds will have to be
improved dramatically, probably using techniques from Arakelov
geometry.


\item
{\em Computation of other BSD invariants, e.g., torsion subgroups,
regulators, and Tamagawa numbers.} In genus 2, many of these questions
are addressed in existing literature (see for instance
\cite{cassels-flynn}) but are not fully implemented. The computation
of torsion subgroups is implemented in Magma. To compute Tamagawa
numbers, one needs the configuration of the special fibers of a
regular model (see (\ref{hyper:euler}) above) and how Galois acts
on the components. Given a finite index subgroup, one can compute the
regulator (up to a rational square) using algorithms due to M\"uller
\cite{mueller:computing} (implemented in Magma) or Holmes
\cite{holmes:computing}. For $g=2$ (and soon for $g=3$ due to work in
progress of Stoll) one can compute the regulator exactly using
Magma. See (\ref{hyper:euler}) above. 

\item
{\em Computation of $p$-adic regulators and comparison with special
  values of $p$-adic $L$-functions.}  Balakrishnan and Besser
\cite{balakrishnan-besser} describe a method to compute $p$-adic
heights via Coleman integration. Putting this together with work of
M\"{u}ller \cite{mueller:computing} gives us a means of computing
$p$-adic regulators. Pollack and Stevens
\cite{pollack-stevens:overconv} gave an algorithm that computes
special values of $p$-adic $L$-functions to high precision using
overconvergent modular symbols.  Balakrishnan, M\"{u}ller, and Stein
\cite{bms} recently stated a $p$-adic analogue of the Birch and
Swinnerton-Dyer conjecture for modular abelian varieties and
numerically compared $p$-adic special values to $p$-adic regulators,
thus giving evidence for the conjecture.  PI Stein's Ph.D. student
Simon Spicer is doing his thesis work on optimization and applications
of the overconvergent modular symbols algorithm, which would be partly
supported by this grant.



\item {\em Determination of all curves of a given genus with conductor
in a given range, assuming appropriate modularity conjectures.} This
is probably only feasible for genus 2, and in any case is closely
related to Siegel modular forms (see Section~\ref{sec:siegel}).  For
small conductors we can make use of $L$-function techniques (see
Section~\ref{sec:hyperelliptic1000}).

\item {\em Rational points.} The most promising methods for this are
variants of the Chabauty method, possibly combined with the
Mordell-Weil sieve, see for instance \cite{bruin-stoll:mwsieve}. Both
of these depend on (partial) knowledge of the Mordell-Weil group. In
genus 2, this should cover most cases encountered in practice; see for
example \cite{bruin-stoll}.

\item
{\em Sato-Tate distributions (or equivalently motivic Galois groups).}
These are fully classified in genus 2; see
\cite{fite-kedlaya-rotger-sutherland}.  We will consider genus~$3$ next.

\end{enumerate}

\subsection{Hyperelliptic 1000}


\label{sec:hyperelliptic1000}

At present there exist extensive tables of elliptic curves$/\Q$ which
are both accurate and complete~\cite{cremona}.  These tables are made
possible by three theoretical results: the fact that elliptic curves
are modular~\cite{breuil-conrad-diamond-taylor}, the fact
that there exists a method (modular symbols) for finding a basis of
the space of holomorphic cusp forms of a given weight and level, and
also the fact that one can directly construct the elliptic curve from the
modular form.

For hyperelliptic curves, none of those results are currently known.
In fact, it was only recently~\cite{BrumerKramer} that the proper modularity
conjecture was even formulated, indicating precisely which weight~2
Siegel modular forms should be associated to hyperelliptic curves of
genus~2. % This refined a more general conjecture of Yoshida.

%Developing an analogue of modular symbols for 
Developing new ways to generate
Siegel modular forms is
a major undertaking of this proposal (see Section \ref{sec:siegel}).
But before such results are available, there is another approach,
using $L$-functions, which we can use to develop a provably complete
table of hyperelliptic curves of small conductor.

Some progress in this direction is already known:

\vspace{0.05in}
\noindent{\bf Theorem}[Brumer and Kramer] 
\emph{
Suppose $A$ is a semistable abelian surface of odd non-square
conductor N.
If $N \le 500$, then $N$ can only be
249, 277, 295, 349, 353, 389, 427, 461, for which examples are known,
 or 415, 417, which we expect do not occur.
}
\vspace{0.05in}

Thus, at least for odd conductors, the list of known hyperelliptic
curves is complete, at least up to $413$.  The reason that Brumer and
Kramer cannot resolve the case of $N=415$ is that their method for
eliminating a conductor (for which they suspect there is no
corresponding hyperelliptic curve) is to show the nonexistence of a
number field with certain properties.  But for $N=415$ there
does exist such a field, but presumably no corresponding hyperelliptic
curve, so their method fails.

We propose using analytic $L$-function techniques to
handle some cases which are not accessible by algebraic methods,
leading to a provably complete table of hyperelliptic curves of small conductor.
This is an illustration of our theme that new results follow from
bringing together different areas.

A genus 2  hyperelliptic  curve has a degree 4 $L$-function which
(in the analytic normalization) satisfies a functional equation
of the form
\begin{equation}
\label{eqn:hyperL}
\Lambda(s) = N^{s/2}\Gamma(s+\tfrac12)^2 L(s) = \pm \Lambda(1-s),
\end{equation}
where $N$ is the conductor of the curve.
In that normalization, the Dirichlet coefficients are of the form
$a_p = A(p)/\sqrt{p}$ where $A(p)$ is a rational integer with
$|A(p)|\le 4\sqrt{p}$.


Work of PI Farmer, together with Koutsoliotas and Lemurell, 
shows
\vspace{0.05in}
\noindent{\bf Proposition}[FKL]  \emph{There is no $L$-function satisfying \eqref{eqn:hyperL}
with $N=415$,
with Dirichlet coefficients $a(p^j) = A(p^j)/p^{j/2}$ where
$A(p^j)$ are rational integers satisfying $|A(p)|\le 4\sqrt{p}$ and
$|A(p^2)|\le 10 p$.
}
\vspace{0.05in}

Thus, assuming a modularity conjecture, there is no hyperelliptic
curve of conductor~$415$.

The idea behind the calculation is that one can use the approximate
functional equation to write down equations for the Dirichlet coefficients,
as in Section~\ref{sec:Lsearch}. But instead of solving
the system, one checks the consistency of the system using the
(finitely many) possibilities for the first few coefficients and
the Hasse bound for the remaining coefficients. This eliminates
some of the options for the first few coefficients. Then one proceeds
to search the tree of possibilities for the later coefficients, pruning
branches when the initial choices are incompatible with the Hasse bound
for the later coefficients. If such an $L$-function exists, then this method
finds its first several Dirichlet coefficients. If no such $L$-function
exists,
% and one can accurately approximate the $L$-function using
%a small number of Dirichlet coefficients, 
then one can hope that every branch of the
tree is pruned and  the non-existence becomes a theorem.

The above proposition is work-in-progress, but based on our experience in that
calculation, we expect
to determine the
complete list of hyperelliptic curve $L$-functions up to conductor~1000,
and probably a bit further.  This would bring the table of
hyperelliptic curves up to the point where elliptic curve tables were
around 1990.  Considering the enormous research benefits (both direct
and indirect) that have come from making detailed tables of elliptic
curves, we believe that producing such tables and making the data
public is an important research contribution.


\section{Elliptic curves over number fields}
\label{sec:elliptic}

In 2012, J. Cremona successfully enumerated all elliptic curves over
$\Q$, ordered by conductor, up to the first curve of rank $4$ (which
has conductor 234,446). This major milestone was the
culmination of over two decades of progress, including
improvements in theory and implementations of algorithms and access to
more powerful computing equipment.  Inspired by Cremona's work, we
propose to carry out similar projects over number fields other than
$\Q$.  Work in this section will be coordinated by PIs Gunnells and
Stein, with support from senior scientists Ash and Cremona.

Our initial work, described in Section~\ref{sec:ECQrt5}, will focus on
quadratic fields, and more specifically on real quadratic fields.
Focusing on totally real number fields $F$  %, in which all embeddings
%$F\hookrightarrow \C$ land in $\R$, 
grants us substantial extra
structure.  In particular, there is a {\em conjectural} bijection
between isogeny classes of elliptic curves over $F$ and rational
Hilbert modular newforms of parallel weight 2 (this bijection is only
known when $F=\Q$ -- see \cite{breuil-conrad-diamond-taylor}), and
this bijection provides functional equations for $L$-functions,
Heegner points, and results toward the Birch and Swinnerton-Dyer
conjecture (see \cite{zhang:heightsshimura}).

One missing ingredient is an efficient algorithm to construct the
(conjectured) elliptic curve from the Hilbert modular form.  We are
forced to combine a variety of strategies, described below, which in
the course of this project we hope to make more efficient.

We also consider elliptic curves over nonreal cubic fields in
Section~\ref{sec:ECcubicfields}.  For that case, the underlying theory
is less developed, and there is a significant amount of basic research
to be done.  


\subsection{Elliptic curves over $\Q(\sqrt{5})$}
\label{sec:ECQrt5}
Our goal is to find all modular elliptic curves over $\Q(\sqrt{5})$,
ordered by conductor, up to the first of analytic rank 4.

Noam Elkies found the smallest known conductor of a elliptic curve of
rank $4$ over $\Q(\sqrt{5})$ that is not a base change from $\Q$; it
has norm conductor 1,209,079.  We estimate that it should be possible
to find all the rational Hilbert modular newforms of norm conductor up
to 1,209,079 in around 200,000 hours of CPU time, e.g., less than two
months using the hardware we are requesting as part of this proposal.
It is difficult to estimate the time it will take to find the
corresponding elliptic curves, since there is no known efficient
algorithm to find an elliptic curve attached to a newform (in some
cases, we do not even know there is a curve, though this is
conjectured).

We use an optimized version of the algorithm in \cite{dembele:hilbert5} to
compute all rational Hilbert modular cusp forms over $\Q(\sqrt{5})$ of
weight $(2,2)$ and level $\n$.  In order to implement this algorithm,
it is critical that we can compute with $\PP^1(R/\n)$ very, very
quickly; as explained in \cite[\S2.2]{sqrt5first} we have a highly
optimized implementation that we have customized for the case
$\Q(\sqrt{5})$.  Supported by this grant, PI Stein's graduate student
R.\thinspace{}A. Ohana intends to generalize this approach to other
fields.

%[[discussion of computing rational hilbert newforms; basically what we did at snowbird]]

Some techniques for finding a curve attached a
newforms $f$ with Hecke eigenvalues $a_{\pp}(f)$, for primes $\pp$ of the
ring of integers of $\Q(\sqrt{5})$:
\begin{enumerate}
\item {\em Stein-Watkins search} -- consult a database made by doing a search
in the style of \cite{stein-watkins:ants5}.
\item {\em Sieved enumeration} -- use $a_{\pp}(f)$ to impose congruence conditions on the coefficients 
of the Weierstrass equation, then search.
\item {\em Torsion families} -- use $a_{\pp}(f)$ to determine whether
  $\ell\mid \#E(F)$ for some $E$ attached to $f$, and if so search
  for $E$ in the family of curves with $\ell$-torsion.
\item {\em Congruence families} -- if we know some $E'$ and that $E'[\ell]\ncisom E[\ell]$, use
Tom Fisher's explicit families \cite{fisher:families_cong}.
\item {\em Twisting} -- find a minimal conductor twist $f^{\chi}$ of $f$, find a curve
attached to $f^{\chi}$, then twist it to get a curve attached to $f$.
\item {\em Cremona-Lingham} -- find many curves with good reduction outside the level $\nn$ of $f$ by searching for integral points on auxiliary curves \cite{cremona-lingham}.
\item {\em Dembele, Bober} -- determine periods from special values of $L$-series \cite{dembele:elliptic-curves-quadratic-fields}.
%\item {\em Elkies} -- use the $\lambda$ inv
\end{enumerate}

Once we find a curve in the isogeny class corresponding to the
rational Hilbert newform, Billerey's algorithm \cite{billerey}
combined with Velu's equations for isogenies enable us to explicitly
compute representative elliptic curves for each isomorphism class.
%Billerey's algorithm does not depend on any conjectures, and works
%over any number field.

In \cite{sqrt5first}, the PI Stein and numerous undergraduates,
graduate students, and a postdoc, made a complete table of every
elliptic curve over $\Q(\sqrt{5})$, up to the first curve of rank $2$
(which has norm conductor $1831$).  Stein intends to partly support
one of these graduate students using this grant (A. Deines) to carry
out a Stein-Watkins \cite{stein-watkins:ants5} style search for curves
over $\Q(\sqrt{5})$.  Another one of the co-authors (R.A. Ohana)
developed the sieved enumeration method, and is starting graduate
school now at UW; Stein intends to support Ohana to do further work in
this direction.

%This work will also produce a multitude of Hilbert modular forms, which
%are of independent interest.  Stein will work with LMFDB collaborator Voight
%to make this data available.

\subsection{Elliptic curves over complex cubic fields}
\label{sec:ECcubicfields}

In this section we describe the computation of tables of elliptic curves over
complex cubic fields $F$ (i.e., cubic fields with exactly one real
place).  Such computations have already been carried out for $F$ the
cubic field of discriminant $-23$ in work of PI Gunnells with Yasaki
and Klages-Mundt \cite{complexcubic, neg23} up to conductors of norm
911.  Our goals are to extend these computations in several
directions. See Section~\ref{sec:CohoArithGps} for background. 

As in Section \ref{sec:ECQrt5}, one expects that for every elliptic
curve $E$ over $F$, there should exist a cuspform $f$ on $\GL_{2}/F$
with rational Hecke eigenvalues such that $L (s,f) = L (s,E)$.  Hence
the first step is computing a table of suitable automorphic forms on
$\GL_{2}/F$, so that one has a (conjectural) list of conductors of
possible curves.  Note that the connections between such forms and
arithmetic are completely unproven: one doesn't even know, for
example, that such forms will have $\ell$-adic families of Galois
representations attached to them, in contrast to the Hilbert modular
case. 

To compute the relevant automorphic forms, we use the fact that they
are cohomological and compute the
cohomology of subgroups of $\GL_{2} (\sO)$, where $\sO$ is the ring of
integers of $F$.  These methods are quite different than those 
described in Section~\ref{sec:ECQrt5} and we refer to Section~\ref{sec:CohoArithGps}
for background.
For any ideal $\n\subset \sO$, we
consider the subgroup $\Gamma_{0} (\n)$ of matrices upper-triangular
mod $\n$ and form the locally symmetric space $Y_{\n} = \Gamma_{0}
(\n) \backslash G/KA_{G}$.  The space $Y_{\n}$ has dimension $6$, the
virtual cohomological dimension $\nu$ is $5$, and the top of cuspidal
range is $4$.  There is an explicit reduction theory due to Koecher
\cite{koecher}, generalizing Voronoi's work on perfect quadratic forms
\cite{voronoi}, that allows us to construct a cell decomposition of
$Y_{\n}$ for any $\n$.  Using this we compute $H^{4} (Y_{\n} ;\C) $
for all ideals up to some bound on their norm.  Note that work of Ash
\cite{ash.minimal} provides an analogue of the classical theory of
modular symbols for $\GL_{2}/\Q$ \cite{manin} for this setting, but it
computes $H^{5}$, not $H^{4}$, and thus does not see the cuspidal
cohomology.

To identify the forms corresponding to elliptic curves, we must
compute the Hecke action on $H^{4}$.  This requires some effort, since
even though the Hecke operators act on cohomology, they don't preserve
the Koecher cells, and thus do not act directly on the cochain complex
built from these cells.  The classical theory of modular symbols
encounters the same problem.  More precisely, consider the Satake
compactification $\fH^{*}$ of the upper halfplane, suppose $\Gamma
\subset \SL_{2} (\Z)$ is a congruence subgroup, and let $X_{\Gamma} =
\Gamma \backslash \fH^{*}$.  Any two cusps $u,v$ determine a modular
symbol $[u,v]\in H_{1} (X_{\Gamma }, \partial X_{\Gamma }; \C)$: one
takes the class of the image of the oriented ideal geodesic in
$\fH^{*}$ running from $u$ to $v$.  The analogues of the Koecher cells
are the $\SL_{2} (\Z)$-translates of the geodesic from $0$ to $\infty$;
the resulting modular symbols are called unimodular symbols.  One can
determine the relations among the modular symbols to determine a
combinatorial model for the relative homology, which gives a model for
$H^{1} (X_{\Gamma}\smallsetminus \partial X_{\Gamma }; \C)$ by
duality.  The modular symbols admit an action of the Hecke operators
compatible with duality, but the action does not preserve the subspace
of unimodular symbols.  

In the setting of $\GL_{2}/F$, the Koecher cochain complex is the
analogue of the space of unimodular symbols, and to compute the Hecke
action we embed it in a larger complex, the \emph{sharbly complex}.
This complex, which gives a resolution of the Steinberg module of
$\GL_{n} (F)$, is essentially the chain complex of the free simplicial
complex on the cusps of a certain compactification of $G/KA_{G}$ %,
%although there are some subtleties 
\cite{heckants, complexcubic,
AGM5}.  Computing the Hecke action involves %is then done by an algorithm that
taking a sharbly cycle representing a class in $H^{4} (Y_{\n}; \C)$ and
rewriting it in terms of cycles supported on Koecher chains.  This
is similar to the approach used in the context of $\SL_{4} (\Z)$
\cite{experimental, computation, AGM2, AGM3, AGM4, AGM5}, another
situation in which the top of the cuspidal range is $\nu -1$.  This
relies heavily on a generalization of the Ash--Rudolph
algorithm \cite{ash.rudolph} for modular symbols on $\GL_{n} (\Z)$.
Computing Hecke operators is the main bottleneck, although one can
often predict the existence of elliptic curves at higher level norms
by computing ranks of cohomology groups and using a yoga of ``old and
new'' cohomology classes \cite{torsioncubic}.
%  For instance, there may
%be a level at which all of $H^{4}$ is accounted for by Eisenstein
%series and old cohomology classes, except for one dimension; this
%predicts the existence of a rational eigenclass and thus an elliptic
%curve of this conductor.

Once one has a list of possible conductors, one must find equations of
elliptic curves in each isogeny class.  To do this we use similar
techniques as in Section \ref{sec:ECQrt5}: naive enumeration,
Cremona--Lingham, searching in torsion families, and so on. 
Stein and Gunnells will coordinate on this work.
% These
%methods suffer from the same drawbacks as in the $\Q(\sqrt{5})$ case,
%although they are remarkably useful in practice.  
Once a curve in each
isogeny class is found, work of Billerey \cite{billerey} enables us to
compute representatives for each of the isomorphism classes, just as
for the $\Q(\sqrt{5})$ case.

We propose to work towards the following goals:

\begin{enumerate}
\item Extension of the tables in \cite{complexcubic, neg23} as far as
possible.
\item Extending the computations in \cite{complexcubic, neg23} for other
complex cubic fields. % Some %preliminary 
Data for elliptic curves over
the fields of discriminants  $-31$, \dots , $-107$ can be found at
\cite{ariah} and in \cite{torsioncubic}.
\item Extend the computations of \cite{complexcubic} to nontrivial
coefficients, i.e.~compile tables of higher weight cuspforms.
\item Incorporate all this data into the LMFDB.
\end{enumerate}






\subsection{Already in intro?}
We propose a workshop in which each day features two talks providing
background information and describing recent work and current problems
of interest.  Each afternoon will involve group work in which
the participants are actively engaged in a variety of activities related
to the workshop focus.  Those activities will include starting or
continuing research projects, going through the details of advanced
material which was not covered in the lectures, and working to put
new material into the $L$-functions and Modular Forms Database 

\end{comment}

\section{Funding}

All requested funds are for participant support.  Because of the
workshop side of this proposal, we hope to invite $30$ people with
funding to cover their travel, food, and lodging.  

The conference will be open to all mathematicians.
After funding invited participants, priority 
follows the reverse seniority rule: highest priority goes to graduate
students, then post-docs and junior faculty, and finally to established
faculty who do not have external funding to support their travel.
Mathematicians from underrepresented groups are treated one level
higher in terms of priority.

% hotel $110 per night

The School of Mathematical and Statistical Sciences at ASU has pledged to
contribute \$7,000  to help support this conference.
Since ASU is an IMA participating institution, we will apply to the
IMA for a \$5,000 conference grant as well.

\section{Logistics}

The conference will be held on the main campus of Arizona State
University in Tempe, AZ.  Tempe can be easily accessed from Phoenix
Sky Harbor international airport, which is a hub for a major airline,
so is easily accessible.

ASU will provide meeting space and wifi access for conference
activities, and the School of Mathematical and Statistical Sciences at
ASU will provide secretarial and clerical support for the conference.

\subsection{Personnel}

The organizing committee for the conference is David
Farmer, Paul Gunnells, 
John Jones, and Holly Swisher.

We have tentative agreements from the following mathematicians to give
plenary talks:
John Cremona, Kiran Kedlaya, Sally Koutsoliotas, Stefan Lemurell, 
Cris Poor, William Stein, Andrew Sutherland, John Voight, and David Yuen.

\subsection{Daily organization}


Each day will feature two talks in the morning providing background
information and describing recent work and current problems of
interest.  Each afternoon will involve group work in which the
participants are actively engaged in a variety of activities related
to the workshop focus.  Those activities will include starting or
continuing research projects, going through the details of advanced
material which was not covered in the lectures, and working to put new
material into the $L$-functions and Modular Forms Database
(\url{http://www.LMFDB.org}).

\subsection{Dissemination}

Dissemination of results from this workshop will be through the
LMFDB.  This web site is the result of an ongoing international
collaboration, with contributions from roughly $50$ number theorists
thus far.  It serves the mathematical community in two ways.

First, it provides results of rigorous computations of mathematical
objects of interest in modern number theory organized around the theme
of $L$-functions.  The objects include elliptic curves over the
rationals, number fields, and assorted types of automorphic forms
(classical, Hilbert, Siegel).  Computational data of this sort has
proven useful to number theory researchers in formulating and testing
conjectures, and in some cases can play a role in the proof of
theorems.

Second, the LMFDB plays an educational role for people interested in
number theory.  It has an overview of number theoretic objects,
contains brief explanations regarding the objects it contains, and
shows interconnections between them.  

For example, when viewing the web page for a specific elliptic curve
over $\Q$, the user sees the important invariants of the curve such as
its conductor, rank, and Mordell-Weil generators.  He will also have
links to pages for its associated modular form, its $L$-function, and
symmetric powers of its $L$-function.

During the afternoons, we will have partipants extending and improving
the LMFDB in ways related to the themes of this workshop.  We expect
to expand the LMFDB to include sections for hyperelliptic curves over
$\Q$, elliptic curves over various quadratic fields, and their
associated $L$-functions.  We also expect to improve the areas on
Hilbert and Siegel modular forms, and to improve the
presentation/expository aspects of the site.



\begin{comment}
Recent advances to be covered in workshop lectures:

\subsection{Elliptic curves over $\Q(\sqrt{5})$}

William Stein and collaborators have been developing the analogue of
Cremona's tables for elliptic curves over $\Q(\sqrt{5})$.  There are
numerous difficulties which are not present in the rational case:
modularity is not known, given the modular form, the construction of
the corresponding elliptic curve is not known, etc.

One or two lectures by Stein.

\subsection{Hilbert modular forms}

John Voight has made extensive tables of Hilbert modular forms,
primarily over $\Q(\sqrt{5})$ and $\Q(\sqrt{8})$.  The basic
theory, the computational methods, and the associated $L$-functions
(which provide the connection to elliptic curves) will be
described.

One or two lectures by Voight.

\subsection{Hyperelliptic curves}

Kiran Kedlaya and Andrew Sutherland have produced a wide variety
of examples of hyperelliptic curves (primarily genus $2$, but 
genus $3$
calculations will be available by the time of the workshop).

One lecture each by Sutherland and Kedlaya

\subsection{Paramodular forms}

Siegel modular forms on the paramodular group are (conjecturally) the
modular objects associated to genus 2 hyperelliptic curves.
Cris Poor and David Yuen have produced the first examples of these
objects.  These calculations are significantly more difficult than
the corresponding work on Hilbert modular forms.

One lecture by Poor or Yuen

\subsection{Elliptic curves over $\Q(i)$}

Elliptic curves over imaginary quadratic fields are fundamentally
different than elliptic curves over real quadratic fields, primarily
because the associated modular object is a Bianchi modular form,
not a Hilbert modular form.  Work in this area is much less advanced
(despite the fact that John Cremona made tables of elliptic curves
over $\Q(i)$ in his thesis.

One lecture by Cremona

\subsection{$L$-function techniques for hyperelliptic curves}

Current methods of producing tables of hyperelliptic curves are not
capable of proving that the list is complete.  David Farmer, Sally
Koutsoliotas, and Stefan Lemurell and an L-function approach which
(assuming modularity) can provide the missing step in the proof.

One lecture by Farmer, Koutsoliotas, or Lemurell.
\end{comment}

\begin{thebibliography}{999}

\bibitem{sqrt5first}
J.~Bober, A.~Deines, A.~Klages-Mundt, B.~LeVeque, R.~A. Ohana, A.~Rabindranath,
  P.~Sharaba, and W.~Stein, \emph{A {D}atabase of {E}lliptic {C}urves over
  {$\mathbf{Q}(\sqrt{5})$}---{F}irst {R}eport}, Proceedings of the 10th
  International Symposium (ANTS-X) (2012).

\bibitem{BrumerKramer}
A.~Brumer and K.~Kramer, \emph{Paramodular abelian verieties of odd conductor},
arXiv:1004.4699.


\bibitem{cremona1}
J.~E.~Cremona, \emph{Hyperbolic tessellations, modular symbols, and elliptic curves over complex quadratic fields},
Compositio Math. \textbf{51} (1984), no.~3, 275--324.

\bibitem{cremona-lingham}
J.\thinspace{}E. Cremona and M.\thinspace{}P. Lingham, \emph{Finding all
  elliptic curves with good reduction outside a given set of primes},
  Experiment. Math. \textbf{16} (2007), no.~3, 303--312. 

\bibitem{cremona}
J. E. Cremona and E. Whitley, {Periods of cusp forms and elliptic curves over imaginary quadratic fields},
Math. Comp. {\bf 62} (1994), no. 205, 407--429.

%\bibitem{fite-kedlaya-rotger-sutherland}
%F. Fit\'e, K.S. Kedlaya, V. Rotger, and A.V. Sutherland, Sato-Tate distributions and Galois endomorphism
%modules in genus 2, \textit{Compositio Math.} (published online 2012).


\bibitem{quartic}
P.~E. Gunnells, F.~Hajir, and D.~Yasaki, \emph{Modular forms and elliptic
  curves over the field of fifth roots of unity}, to appear in Exp.~Math.

\bibitem{neg23}
  P.~E. Gunnells and A.~Klages-Mundt, \emph{Modular elliptic curves over the
  cubic field of discriminant $-23$}, In preparation.

\bibitem{complexcubic}
 P.~E. Gunnells and D.~Yasaki, 
 \emph{Modular forms and elliptic curves over the
    cubic field of discriminant $-23$}, Int. J. Number Theory, \textbf{9},
 (2013), no.~1, 53--76.

\bibitem{KS-1} K.S.~Kedlaya and A.V.~Sutherland,
Hyperelliptic curves, $L$-polynomials, and random matrices, in
\textit{Arithmetic, Geometry, Cryptography, and Coding Theory (AGC$^2$T
2007)}, Contemp. Math. 487, Amer. Math. Soc., 2009.

\bibitem{kedlaya-sutherland-ants}
K.S.~Kedlaya and A.V.~Sutherland,
{\em Computing $L$-series of hyperelliptic curves},
in Algorithmic Number Theory (ANTS VIII),
Lecture Notes in Comp. Sci. 5011, Springer, 2008, 312--326.

\bibitem{poory}
C.~{Poor} and D.~S. {Yuen}, 
\emph{Computations of spaces of Siegel modular cusp forms},
J. Math. Soc. Japan \textbf{59} (2007), no.~1, 185--222. 

\bibitem{PY_para_p}
C.~{Poor} and D.~S. {Yuen}, 
\emph{{Paramodular Cusp Forms}}, ArXiv e-prints
  (2009).


\end{thebibliography}

\end{document}

