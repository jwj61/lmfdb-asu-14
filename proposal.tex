\documentclass[12pt]{amsart}

\setlength{\topmargin}{-0.3in} 
\setlength{\textheight}{8.2in}
\setlength{\textwidth}{6.5in} 
\setlength{\oddsidemargin}{0in}
\setlength{\evensidemargin}{0in}
\addtolength{\footskip}{0.2in}
\renewcommand{\baselinestretch}{1.4}
\addtolength{\parskip}{0.3ex}
\renewcommand{\bibliofont}{\normalsize}

\usepackage{datetime}
%\usepackage{amsmath}
\usepackage{graphics}
\usepackage{verbatim}

\usepackage{url}
\usepackage{amssymb}
\usepackage{mathrsfs}
\usepackage{cite}
\DeclareMathOperator{\SL}{SL}
\DeclareMathOperator{\GL}{GL}
\DeclareMathOperator{\PGL}{PGL}
\DeclareMathOperator{\Sp}{Sp}
\DeclareMathOperator{\GSp}{GSp}
\DeclareMathOperator{\USp}{USp}
\newcommand{\pp}{\mathfrak{p}}
\newcommand{\nn}{\mathfrak{n}}
\newcommand{\p}{\mathfrak{p}}
\newcommand{\n}{\mathfrak{n}}
\newcommand{\ncisom}{\approx}   % noncanonical isomorphism

\pagestyle{plain}
% Linked CONTENTS
%\usepackage{hyperref}
%%%%%%%%%%%%%%%%%

%\setlength{\parskip}{2ex}
%\setlength{\parskip}{0.1ex}

% \voffset -0.5in
% \hoffset -0.8in
% %\oddsidemargin  -0.3125in
% %\evensidemargin -0.3125in
% \textwidth      6.475in
% \headheight     0.0in
% \topmargin      0.063in
% \textheight=9.3875in

\numberwithin{equation}{section}

% ---- SHA ----
%\DeclareFontEncoding{OT2}{}{} % to enable usage of cyrillic fonts
%  \newcommand{\textcyr}[1]{%
%    {\fontencoding{OT2}\fontfamily{wncyr}\fontseries{m}\fontshape{n}%
%     \selectfont #1}}
%\newcommand{\Sha}{{\mbox{\textcyr{Sh}}}}
%\newcommand{\cyr}{\textcyr}
\newcommand{\nc}{\newcommand}
\newcommand{\C}{\mathbb C}
\newcommand{\Q}{\mathbb Q}
\newcommand{\QQ}{\mathbb Q}
\newcommand{\PP}{\mathbb P}
\newcommand{\Z}{\mathbb Z}
\newcommand{\R}{\mathbb R}
\newcommand{\gal}{\textrm{Gal}}

% from PG
\newcommand{\bG}{\mathbf{G}}
\newcommand{\sO}{\mathscr O}
\newcommand{\sM}{\mathscr M}
\newcommand{\bP}{\mathbf{P}}
\newcommand{\F}{\mathbb{F}}
\newcommand{\fH}{\mathfrak{H}}
\newcommand{\cusp}{\text{cusp}}

\title{Conference: Curves and Automorphic Forms}
\begin{document}
%\begin{center}
%   \settimeformat{ampmtime}\bf \Large \today{} at \currenttime{}
%\end{center}
%(Temporary table of contents so we can find our way around.)
%\tableofcontents\vfill\mbox{}
%
%\break


\section{Overview}

The theme for this conference is connections
between curves, modular forms, $L$-functions, and their computational
aspects.
Our goals are to provide a setting to
\begin{itemize}
\item allow dissemination of recent research results;
\item foster collaboration among experts working in areas which are
  related;
\item aid in the development of number theorists who are early-career
  or from underrepresented groups; and 
\item make relevant computational results publicly available.
\end{itemize}

The format of the event will combine elements of both a conference and
a workshop to achieve its goals.  Each morning will feature two plenary
talks given by invited speakers, starting with background for their
areas and leading to recent results.  Afternoons will be devoted to
participants working in small groups.  Some groups will work on
questions raised by the morning lectures, some will work on
computations, and others will work on presenting computational
results to the public.

Computational results will be made available through the website {\em
  $L$-Functions and Modular Forms Database}
(\url{http://www.lmfdb.org}), referred to as the \textsf{LMFDB}.
This site is an international collaboration dedicated to presenting
computational data in number theory and related fields to assist
mathematicians conducting research and also to help educate
visitors to the site about some of the central objects of modern
number theory and their interconnections.

Computational projects provide a good path for young mathematicians to
develop intuition for an area.  Most of the senior participants will
be well-versed in computational number theory.  We will also have
experts from the \textsf{LMFDB} in attendance to aid people
who may be contributing to that project for the first time.  Since the
\textsf{LMFDB} uses \textsf{Sage}, an open-source computational
system, we will invite a significant number of participants from the
{\em Women in Sage} community to help broaden the representation of
women in this conference.

\bigskip

Theoretical and computational work on elliptic curves over the
rationals has enjoyed great advances over the past few decades.
Recently there has been surge of work on extending this in two
directions, namely to hyperelliptic curves over the rationals and
elliptic curves over quadratic fields.  In addition, there have been
parallel advances on the corresponding automorphic objects, namely
Siegel and Hilbert modular forms.  A major objective of this
conference is to present this recent work to a wider audience and in
the process to clarify the connections between the algebraic and
automorphic objects.  There is a large amount of useful related
numerical data that we would like to make available to the
mathematics research community.

Recent advances to be covered in workshop lectures:
\begin{enumerate}
\item  Elliptic curves over real quadratic fields (one or two lectures).

\noindent
William Stein and collaborators have been developing the analogue of
Cremona's tables for elliptic curves over $\Q(\sqrt{5})$.  There are
numerous difficulties which are not present in the rational case
and also many recent methods for overcoming these difficulties.

\item Hilbert modular forms (one or two lectures).

\noindent
John Voight, in collaboration with Steve Donnelly, 
has made extensive tables of Hilbert modular forms,
including over $200{,}000$ forms over totally real fields of degree at
most $6$ and including 
many forms over quadratic fields.  The basic
theory, the computational methods, and the associated $L$-functions
(which provide the connection to elliptic curves) will be
described.

\item  Hyperelliptic curves (two lectures).

\noindent
Kiran Kedlaya, Andrew Sutherland, and their collaborators have
undertaken a systematic investigation of the Sato-Tate group of 
hyperelliptic curves (primarily of genus $2$, but also some in genus $3$)
and consequently have tabulated a wide variety
of examples.


\item  Paramodular forms (one lecture).

\noindent
Siegel modular forms on the paramodular group are (conjecturally) the
modular objects associated to genus 2 hyperelliptic curves.
Cris Poor and David Yuen have produced the first examples of these
objects.  

\item  Elliptic curves over non-totally real number fields (two lectures).

\noindent
Elliptic curves over imaginary quadratic fields are fundamentally
different than elliptic curves over real quadratic fields, primarily
because the associated modular object is a Bianchi modular form
rather than a Hilbert modular form, and consequently even the
most basic construction of an elliptic curve from a Bianchi form is
still unknown.  Even with these limitations, John Cremona and his
students, and more recently Dan Yasaki, have made tables of elliptic curves
over imaginary quadratic fields such as $\Q(i)$.

%One lecture by Cremona

\item  $L$-function techniques for hyperelliptic curves (one lecture).

\noindent
Current methods of producing tables of hyperelliptic curves are not
capable of proving that the list is complete.  David Farmer, Sally
Koutsoliotas, and Stefan Lemurell have an $L$-function approach which
(assuming modularity) can provide the missing step in the proof.

\end{enumerate}

\section{Scientific content}

\subsection{Hyperelliptic curves over $\Q$}

In this workshop we will focus on recent work concerning hyperelliptic 
curves
of genus~$2$, that is, a smooth curve of the form
\begin{equation}
y^2=f(x)
\end{equation}
where $f\in k[x]$ is a polynomial of degree 5 or~6 and $k$ is a number
field.  Our specific focus is on the recent work of Kedlaya and
Sutherland and their collaborators on the distribution of Euler
factors of the $L$-function of the curve
\cite{KS-1,kedlaya-sutherland-ants}.

Via their Jacobians, this work concerns abelian surfaces; at the
workshop we will bring out the connections with other objects.
Indeed, it is not the hyperelliptic curve itself but rather its
Jacobian $A$ to which we can associate an $L$-function.  There are (up
to conjugacy) $52$ subgroups of $\USp(4)$ that determine the
distribution of Euler factors of the $L$-function of $A$.  That is,
for each abelian surface $A$ there is a closed subgroup of $\USp(4)$,
called the Sato-Tate group of $A$, such that the local factors of the
$L$-function of $A$ have the same limiting distribution as the
characteristic polynomials of matrices in the subgroup.  Kedlaya and
Sutherland exhibit hyperelliptic curves whose Jacobians give examples
of all $52$ cases, of which $34$ occur for $k=\Q$.

Workshop lectures will describe how the Sato-Tate group arises from
the Galois action on the Tate module of~$A$ and how the subgroups were
classified.  While some of this material can be described for any
number field, we will primarily consider hyperelliptic curves over the
rationals.  In this case the $L$-function has degree~4, which makes
computation more tractable, and there are connections to objects from
other lectures, such as Siegel modular forms.  The role of the
Sato-Tate group in identifying the associated modular object and
determining its properties will be the starting point for discussions
on future avenues of research.


\subsection{Elliptic curves over low degree number fields}

An elliptic curve over a number field $k$ has a $L$-function of
degree~$2d$, where $d=[k:\Q]$.  We will be primarily concerned with
quadratic fields, because of their relation to other objects already
in the \textsf{LMFDB}.

In the case of real quadratic fields, lectures will focus on the
computational challenges faced by Stein and his collaborators as they
tabulate elliptic curves over $\Q(\sqrt5)$ \cite{sqrt5first}.  There
are numerous 
aspects of that work which are hampered by our limited knowledge
of this area; this discussion will be the starting point for research
project involving both theory and computation.
%
The modular object associated to $E/\Q(\sqrt5)$ is a Hilbert modular
form on $\GL_2$ over $\Q(\sqrt5)$, which will be the topic of another
series of lectures.

In the case of number fields which are not totally real, there has
been work led by Cremona on computing elliptic fields over quadratic
imaginary fields \cite{cremona1, cremona-lingham, cremona}.  Here the
associated automorphic forms are
conjectured to be Bianchi modular forms, although neither direction of
the correspondence is known.  More recent work by Gunnells,
Hajir, Klages-Mundt, and Yasaki \cite{complexcubic, neg23,quartic}
has started to look at elliptic curves over larger degree
fields, including nonnormal extensions of $\Q$.

\subsection{Hilbert modular forms}

Hilbert modular forms are a generalization of classical modular forms
(on $\GL_2$) to totally real fields $F$.  Although the associated 
Hilbert modular variety associated to a field and level can in general
have quite large dimension, the Jacquet-Langlands correspondence
implies that one can see the same system of Hecke eigenvalues
on quaternionic forms of $\GL_2$ and thereby work on either
a Shimura curve or with a definite quaternion algebra.

Methods for computing Hilbert modular forms have seen substantial
development in the past five years, and now there are large databases
of forms available.  Lectures on Hilbert modular forms would discuss
their algorithmic aspects, focusing on concrete examples and 
future directions of research.

\subsection{Siegel modular forms}

Currently, algorithmic methods for computing Siegel modular 
forms is quite limited compared to
the computation of either classical or Hilbert modular forms.  For
higher weight there is the possibility of employing cohomological
methods, the analogue of modular symbols for classical holomorphic
modular forms.  However, for the weight 2 case, which is associated
to hyperelliptic curves, this is not an option.  Specifically, we are
interested in weight 2 cusp forms on the level-$N$ paramodular group,
defined as
$$
K(N)=\left\{
\left(\begin{matrix}
  * & N* & * &*\\
* & * & * & */N\\
* & N* &*&*\\
N*&N*&N*&*
\end{matrix}\right)
:
*\in{\Z}\right\}\cap \mathrm{Sp}(4, \Q).
$$
This group is an an analogue of the Hecke congruence group
$\Gamma_0(N)$, in the sense that it is normalized by an analogue of
the Fricke involution.

Current methods, primarily due to Poor and Yuen \cite{poory, PY_para_p}, are
computationally
intensive due to the difficulty of finding a basis for the space of
cusp forms.  The main technique is via descent from higher weights,
and exploiting the ``smearing'' action of Hecke operators acting on
non-cuspforms.  Lectures on Siegel modular forms will briefly cover
background and computational methods, and then focus on the
``paramodular conjecture'' \cite{BrumerKramer}, which gives a higher
rank analogue of 
modularity for elliptic curves.


\subsection{$L$-functions}

We consider $L$-functions of the Selberg class.
Each has a Dirichlet series, functional equation, 
Euler product, and must statisfy few technical conditions.  One view of
$L$-functions is that they are the glue which connects related
mathematical objects.  For example, a geometric object is said to be
modular if it has the same $L$-function as an automorphic form.

In this workshop the general theory of $L$-functions will be
introduced, and this framework will be used to provide a high-level
description of the connections between the other objects which have
been introduced.  There are two topics of current research which will
be presented and then will form the basis of later discussions.  The
first is the recent discovery by Farmer, Koutsoliotas, and Lemurell
that it is possible in many cases to generate the $L$-function of an
object without first finding the object itself.  This method has been
successful for a variety of degree~4 $L$-functions, but it does not
work in all cases.  Since the objects being described in the other
lectures are associated with degree 4 $L$-functions, this topic is
timely and the methods and current limitations will be described
carefully.

The second topic is the fact that high degree $L$-functions (degree 3
or larger) are computationally expensive to evaluate.  This limits our
ability to make detailed studies of, for example, the distribution of
zeros of these functions.  The underlying cause of these computational
problems will be described, leading to discussions for possible ways
to improve the calculations.  One possibility, which is speculative
but worth considering, is that properties of the underlying objects
could be exploited in some way.  This will take advantage of the fact
that both the $L$-functions and their underlying objects will be
discussed in detail at the workshop.

\section{Logistics}

The conference will be held on the main campus of Arizona State
University in Tempe, AZ.  Tempe can be easily accessed from Phoenix
Sky Harbor international airport, which is a hub for a major airline,
so is easily accessible.

ASU will provide meeting space and wifi access for conference
activities, and the School of Mathematical and Statistical Sciences at
ASU will provide secretarial and clerical support for the conference.

\section{Personnel}

The organizing committee for the conference is David
Farmer, Paul Gunnells, 
John Jones, and Holly Swisher.
%
We have tentative agreements from the following mathematicians to give
plenary talks:
John Cremona, Kiran Kedlaya, Sally Koutsoliotas, Stefan Lemurell, 
Cris Poor, William Stein, Andrew Sutherland, John Voight, and David Yuen.

% \section{Daily organization}


% Each day will feature two talks in the morning providing background
% information and describing recent work and current problems of
% interest.  Each afternoon will involve group work in which the
% participants are actively engaged in a variety of activities related
% to the workshop focus.  Those activities will include starting or
% continuing research projects, going through the details of advanced
% material which was not covered in the lectures, and working to put new
% material into the $L$-functions and Modular Forms Database
% (\url{http://www.LMFDB.org}).

\section{Budget}

All requested funds are for participant support.  We plan to invite
$30$ mathematicians who we feel will be able to contribute in
significant ways to the conference.  We are requesting funds to pay
for travel, food, and lodging for these people.

We will invite mathematicians all different stages in their careers:
graduate students, post-docs and other recent Ph.D.s, junior faculty,
and senior faculty (in the last case, primarily to give some of the
plenary talks).  We will also focus on inviting mathematicians from
underrepresented groups.  One such group where we expect to have the
most success is with female mathematicians.  There are natural
connections between the LMFDB project and SAGE development.  One of
the conference organizers has been involved in Women in Sage.  We plan
to devote at least a fourth of the invitations to members of
underrepresented groups.

Attendence at the conference will be open to all mathematicians.  We
are requesting funding to defray travel costs of a limited number of
other participants.  Priority for this funding follows the reverse
seniority rule: highest priority goes to graduate students, then
post-docs and junior faculty, and finally to established faculty who
do not have external funding to support their travel.  Mathematicians
from underrepresented groups are treated one level higher in terms of
priority.

% hotel $110 per night

The School of Mathematical and Statistical Sciences at ASU has pledged to
contribute funds to pay for refreshments during the workshop.
% Since ASU is an IMA participating institution, we will apply to the
% IMA for a \$5,000 conference grant as well.

Since we hope to fund almost all participants, ASU is providing
meeting space and refreshments, we will not charge participants a
registration fee.


\section{Dissemination}

Dissemination of results from this workshop will be through the
\textsf{LMFDB}.  This web site is the result of an ongoing international
collaboration, with contributions from roughly $50$ number theorists,
thus far.  It serves the mathematical community in two ways.

First, it provides results of rigorous computations of mathematical
objects of interest in modern number theory organized around the theme
of $L$-functions.  The objects include elliptic curves over the
rationals, number fields, and assorted types of automorphic forms
(classical, Hilbert, Siegel).  Computational data of this sort has
proven useful to number theory researchers in formulating and testing
conjectures, and in some cases can play a role in the proof of
theorems.

Second, the \textsf{LMFDB} plays an educational role for people interested in
number theory.  It has an overview of number theoretic objects,
contains brief explanations regarding the objects it contains, and
shows interconnections between them.  

For example, when viewing the web page for a specific elliptic curve
over $\Q$, users see the important invariants of the curve such as its
conductor, rank, and Mordell-Weil generators.  They also have links to
pages for its associated modular form, isogenous curves, its
$L$-function, and symmetric powers of its $L$-function.

During the afternoons, we will have partipants extending and improving
the \textsf{LMFDB} in ways related to the themes of this workshop.  We expect
to expand the \textsf{LMFDB} to include sections for hyperelliptic curves over
$\Q$, elliptic curves over various number fields, and their
associated $L$-functions.  We also expect to improve the areas on
Hilbert and Siegel modular forms, and to improve the
presentation/expository aspects of the site.


\section{Results from current and prior NSF support}
PI Jones is currently supported by NSF grant DUE-1226081,
``Collaborative Research: Updating the WeBWorK National Problem
Library''.  The project is in its first year.  Thus far, PI Jones has
made changes to the WeBWorK system to facilitate improvements to the
National Problem Library as proposed, and these have been accepted by
the WeBWorK developers.  The project also involves organizing
workshops to have faculty collaborate on the organization of the
WeBWorK National Problem Library; the first workshop is scheduled for
June 2013.

\begin{thebibliography}{999}

\bibitem{sqrt5first}
J.~Bober, A.~Deines, A.~Klages-Mundt, B.~LeVeque, R.~A. Ohana, A.~Rabindranath,
  P.~Sharaba, and W.~Stein, \emph{A {D}atabase of {E}lliptic {C}urves over
  {$\mathbf{Q}(\sqrt{5})$}---{F}irst {R}eport}, Proceedings of the 10th
  International Symposium (ANTS-X) (2012).

\bibitem{BrumerKramer}
A.~Brumer and K.~Kramer, \emph{Paramodular abelian verieties of odd conductor},
arXiv:1004.4699.


\bibitem{cremona1}
J.~E.~Cremona, \emph{Hyperbolic tessellations, modular symbols, and elliptic curves over complex quadratic fields},
Compositio Math. \textbf{51} (1984), no.~3, 275--324.

\bibitem{cremona-lingham}
J.\thinspace{}E. Cremona and M.\thinspace{}P. Lingham, \emph{Finding all
  elliptic curves with good reduction outside a given set of primes},
  Experiment. Math. \textbf{16} (2007), no.~3, 303--312. 

\bibitem{cremona}
J. E. Cremona and E. Whitley, {Periods of cusp forms and elliptic curves over imaginary quadratic fields},
Math. Comp. {\bf 62} (1994), no. 205, 407--429.

%\bibitem{fite-kedlaya-rotger-sutherland}
%F. Fit\'e, K.S. Kedlaya, V. Rotger, and A.V. Sutherland, Sato-Tate distributions and Galois endomorphism
%modules in genus 2, \textit{Compositio Math.} (published online 2012).


\bibitem{quartic}
P.~E. Gunnells, F.~Hajir, and D.~Yasaki, \emph{Modular forms and elliptic
  curves over the field of fifth roots of unity}, to appear in Exp.~Math.

\bibitem{neg23}
  P.~E. Gunnells and A.~Klages-Mundt, \emph{Modular elliptic curves over the
  cubic field of discriminant $-23$}, In preparation.

\bibitem{complexcubic}
 P.~E. Gunnells and D.~Yasaki, 
 \emph{Modular forms and elliptic curves over the
    cubic field of discriminant $-23$}, Int. J. Number Theory, \textbf{9},
 (2013), no.~1, 53--76.

\bibitem{KS-1} K.S.~Kedlaya and A.V.~Sutherland,
Hyperelliptic curves, $L$-polynomials, and random matrices, in
\textit{Arithmetic, Geometry, Cryptography, and Coding Theory (AGC$^2$T
2007)}, Contemp. Math. 487, Amer. Math. Soc., 2009.

\bibitem{kedlaya-sutherland-ants}
K.S.~Kedlaya and A.V.~Sutherland,
{\em Computing $L$-series of hyperelliptic curves},
in Algorithmic Number Theory (ANTS VIII),
Lecture Notes in Comp. Sci. 5011, Springer, 2008, 312--326.

\bibitem{poory}
C.~{Poor} and D.~S. {Yuen}, 
\emph{Computations of spaces of Siegel modular cusp forms},
J. Math. Soc. Japan \textbf{59} (2007), no.~1, 185--222. 

\bibitem{PY_para_p}
C.~{Poor} and D.~S. {Yuen}, 
\emph{{Paramodular Cusp Forms}}, ArXiv e-prints
  (2009).


\end{thebibliography}

\end{document}

